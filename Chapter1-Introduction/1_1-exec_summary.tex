\noindent Physical impairments and mobility hindrances have existed for millennia, and just as long have rollators or walkers, canes, and even guide dogs been implemented as an aid to those in need. Still, none of them are foolproof methods, as accidents continue to occur. In fact, many times, users become disinclined to utilize their mobility aids after they fail them, or prove to be cumbersome.\\

\noindent In light of the problems at hand, this project is an effort to increase the utility, safety, and efficiency of a rollator in order to address not only the growing need of mobility aids, but also the increasing consumer demand for smart solutions added to everyday products.\\

\noindent FORWARD (Feedback Oriented Routing and Walker Assistance with Responsive Direction) is an assistive walker enhanced with guidance algorithms, real-time environmental notifications, adaptive velocity control, and numerous other upgrades designed to extend the mobility, safety, comfort, and confidence of its users. By harnessing the power of sensors, motors, and microprocessors, FORWARD brings the latest of electrical and computer technology to those who are in need, so that walkers can become more beneficial for not only the traditional user, but for those who are sensorially impaired.\\

\noindent Presently, FORWARD is outfitted with continuous range detection for obstacles, neural-network based obstacle categorization, and responsive haptic feedback. The system is already smart and destined to grow smarter in senior design II, but as of now, there are many technologies already implemented on the rollator including a wireless data network and an object-oriented guidance, navigation, and control software framework. We've demonstrated progress and functionality in a demo video submitted to UCF faculty, and are excited as a team to continue developing and improving the project moving forward.\\


