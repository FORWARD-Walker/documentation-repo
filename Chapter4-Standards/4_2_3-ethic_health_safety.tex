\subsection{Ethical, Health, and Safety Constraints}
\noindent The FORWARD walker is primarily to be used as a tool to help people who have various kinds of impairments and/or disabilities. Since the walker is intended to physically support the user, there are many considerations for ethics, health and safety. As it relates to safety, we must not hinder the structural stability of the walker. The chassis must be strong enough to uphold the user and stable enough not to tip. These considerations led to a few of our design constraints, including a minimum weight requirement of 20 pounds and instability tilt detection of 10 degrees. Whenever the walker detects tilting over 10 degrees, this will provide feedback to the DC motors driving the wheels in order that the motors can vary in speed to correct for the instability, depending on the side the walker is leaning towards. These constraints are intended to keep the user from falling if their weight is not distributed properly across the walker or if there is something causing the walker to tip. The majority of the remaining design constraints also relate to safety, but in regard to the motion and feedback of the walker. The FORWARD field of vision, CV model accuracy, and sensor accuracy all are necessary to detect obstacles in the path of the walker. If obstacles are not correctly detected and identified, then the user could be injured. For instance, if the walker does not correctly identify a curb, the user could trip and fall. Another safety constraint is response time. FORWARD must be able to brake to a complete stop within 1 second in case an object encountered that is very close to the walker. If the motors do not brake in time, the walker may collide with an object. Our requirement of feedback latency also relates to safety as the user needs to be aware of their surroundings and how the walker is moving in order to respond appropriately. Too long of a delay in the haptic and audio feedback could cause confusion or disorientation to the user, which may cause them to stumble. Apart from our product, there are ways in which a generic rollator or walker must be operated. Oxford Health has an encompassing pamphlet that describes the safe operation of a generic walker, including how to sit, how to apply the brakes, where to place your feet behind the rollator, and how to proceed upon encountering a curb \cite{oxford_health_2014}.\\

\noindent Health is another important constraint to our design. FORWARD is being developed to assist people with various physical medical conditions, including sensory impairment, blindness, deafness, injury, issues with balance, etc. There is a plethora of purposes for FORWARD as the walker can aid in injury recovery, allow users to gain awareness to their surroundings, and provide stability to the user. Rollators in general can be intended as a means of physical therapy \cite{oxford_health_2014}, and our design simply expands upon this purpose.\\

\noindent Ethics are essential throughout our design in meeting the design constraints and not compromising on the safety of the users. There could be serious ramifications if a larger object, such as a vehicle, is not detected, identified, and avoided accurately. There is therefore a great responsibility required from us to employ the best practices possible to ensure safety. If we choose to forego these constraints or do not utilize quality parts, this product could be a danger both to the user and the people surrounding them. 