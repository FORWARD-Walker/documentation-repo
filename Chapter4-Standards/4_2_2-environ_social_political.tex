% must be safe to operate in public spaces and not be invasive, hence bluetooth earpiece and feedback not delivered out loud. sensors are low profile so as not to 
\subsection{Environmental Constraints}
\noindent It is known that battery manufacturing and disposal is often horrid for the cleanliness of the environment, so the FORWARD project is utilizing rechargeable power banks to avoid burning through batteries. A more pressing concern environmentally is would the conversion of yet another "smart" device be invasive to the enjoyment of public facilities such as nature trails or parks. The answer to that is a no because of how minimalistic the FORWARD system addition to a standard non-enhanced rollator is. The introduction of the electronics chassis, sensors, and bluetooth earpiece should not deter or distract users from enjoying their stroll through nature. Being the motor control is electric, there are no carbon emissions to worry about, and being that the user can abandon the earpiece and rely solely on the haptic feedback for guidance means that there is no risk of not hearing the surroundings at a crucial time because of distraction. Finally, FORWARD not only enhances the viability of safely taking medical rollators out into the environment, but also their viability in the indoor setting. It can be operated in any ambient lighting setting and does not necessitate constant reliance on an external network. Therein lies its passive guidance solely utilizing the range sensors and haptics. The obstacle identification will transmit over Wi-Fi, so it would not be reliable in areas with no network connection. Similarly, the Bluetooth earpiece is limited by its battery life, but that is no environmental or design concern of ours. Increased safety and reduction of collisions due to rollators is an immense benefit to the environment in the respects that it applies to.\\

\subsection{Social and Political Constraints}
\noindent FORWARD is an assistive technology. It is not classified as a wearable one however and it is simply an enhancement made to an existing tool available to the physically impaired, extended to the spatially impaired. This should be viewed by most as a positive advancement, allowing for more participation and mobility in places with obstacles that previously would be difficult for walker users. Given that this is empowering to groups who may or may not have been previously ostracized, FORWARD has a positive impact socially. The main social constraint institutes that FORWARD should help prevent collisions. Such would be a nuisance (if caused by FORWARD's mistake)! Politically, FORWARD does not seem constrained in a considerable amount.\\ 