% intro paragraph
\noindent Chapter 4 discusses the related standards and constraints for both hardware and software used by FORWARD during development and realization. The main consideration for standards is given to connections, medical rollators, and IEEE guidelines on Bluetooth, wireless networks, sensors, motors, and serial communications. In addition, we examine constraints through the lenses of economic, environmental, social, political, ethical, health, and safety.

\subsection{Related Standards}
\noindent FORWARD must comply with all IEEE-related standards relating to sensors, motors, computer vision, and image processing. Taking a brief look at the specifics, none are immediately seen as hindrances to the proceedings of FORWARD’s research. Should an issue arise with IEEE-related standards later in the project, future documentation will indicate that. \\

\noindent IEEE also includes some standards relating to the testing of DC motors (IEEE 113-1985 \cite{ieee113277}). Although we cannot access this standard as it must be purchased, it mentions that there are standards for testing the ripple operations of DC motors as well as their use with rectifier power supplies. This is not a specification that correlates to our design, however, because we are not designing the DC motors themselves. We are simply using motors that have already been designed, tested, and evaluated.\\

\noindent IEEE also has a guide that is provided that gives insight to selecting the correct "valve-regulated lead acid batteries for stationary applications" \cite{ieee11893242}. This would be useful to ponder different factors in selecting the correct power supply. However, again, we are unable to access this document as purchasing is required. Although this document would be helpful, it is not necessary.\\

\noindent The ADA (American Disabilities Act) also has stipulations for environments to make accommodations for these assistive vehicles including walkers and electric wheelchairs. However, none of these should affect our design process, given that our product is safe to operate around other people and in busy environments.\\

\subsubsection{PCB Standards}
\noindent Some standards that may be relevant, however, include the IPC standards for PCB design \cite{ipc2221}. IPC is a trade association that seeks to ensure high-quality products by providing standards for electronics design and manufacturing. The standard IPC-2221 provides specifications for PCB traces and component spacing on the board. We would like to implement at least some of these of these standards so that our PCB can high-quality and reduce the probability of errors in our design. \\ 

\noindent Some of the primary implications of this IPC-2221 standard are the clearance between components on the PCB and the thickness of traces. Different components require different clearances between them; for example, a minimum clearance of 75mm is required between uncoated conductors and other components. The dimensions of the traces on the PCB are also a necessary consideration as current passing through a wire creates heat and may burn out the trace. Thus the thickness of the PCB trace is dependent on the surrounding temperature, width of the trace, maximum rated current, and constants. Below the formula is given \cite{ipc2221}. \\

\noindent \[A = I / ({k \times \Delta T ^ {0.44}}) ^ {1/0.725}\]

\noindent Another implication of the IPC-2221 standard is design considerations for the material of the PCB. A circuit board should have insulating material (at least one layer) between components so that there is electrical isolation and nothing is shorted. It is essential that this material is a good insulator, particularly if the PCB is meant to withstand higher voltages. We plan to use motor drivers within our design for potentially high voltage motors. Thus, we especially need to consider selecting materials that follow the insulation test specifications established by IPC-2221. These are materials with a high CTI (critical tracking index) rating, which have a larger breakdown voltage.\\

%\noindent Another standard that relates to PCB design is 

\subsubsection{Soldering Standards}
\noindent Another IPC standard that we thought to be relevant also relates to the fabrication of the prototype connecting with the PCB in terms of soldering. Due to economic constraints, we are unable to purchase the set of standards defined by IPC J-STD-001J, however, we have access to the table of contents of these soldering standards. This document encompasses over 70 pages of standards related to soldering, some of which include lead forming, unintentional bending, thermal protection, soldering to terminals, wire routing, flux application, and high frequency/voltage applications \cite{ipc_standard}. These are just a handful of topics covered, many of which have some relevance to our project. Although we are incapable of viewing the standards themselves, they still give us an appropriate baseline of applicable factors to consider. From there, we can further research any topics that we believe require attention.\\

\noindent IEC likewise has a standard related to soldering \cite{iec6006869ed2017}. This standard describes the testing and evaluation steps to determine the solderability of components. Although we do not need to implement these standards ourselves as we are not developing integrated circuits, it is important to use components that have passed these standards for soldering. If, for example, we need to solder a pin header to the PCB, we want to be sure that there is a strong connection between the header pins and the board.

\subsubsection{Electromagnetic Compatibility Standards}
\noindent We found another set of standards that again would need to be purchased but could  provide useful insight in designing FORWARD. The International Electrotechnical Commision (IEC) has released the IEC 61000-4-3 which pertains to electromagnetic compatibility (EMC) There are many devices that provide RF or radiation interference that could damage or alter the readings to other devices. To avoid these undesirable effects, these standards require the use of components that have immunity to electromagnetic radiation and RF. These standards are helpful to be aware of because we plan to use a Wi-Fi camera module communicating with the ESP32 through Bluetooth. We do not want to be using components that could easily give incorrect ratings due to unknown interference, so we can also search for components that meet these standards. Although we may not be able to access the IEC standards, some components may list if they are compliant to this particular standard \cite{iec_standard}.\\

\noindent The Federal Communications Commission (FCC) also has standards related to the topic of electromagnetic compatibility. These standards include unintentional radiators and intentional radiators. Our FORWARD design contains both types of radiators, for example, the Bluetooth modules as an intentional radiator and the power supply an unintentional radiator. So long as the device is emitting RF or another EM wave for the purpose of functionality, this device is an intentional radiator. Otherwise, if an EM wave is emitted undesirably or unnecessarily, this device is an unintentional radiator. As the FCC is a federal commission, these standards are not optional [Appendix C] but rather enforced, and thus every component that we purchase should comply with these standards so long as they are legally sold in America \cite{fcc_unintentional_radiators} \cite{fcc_intentional_radiators}.

\subsubsection{Voltage Interruption Standards}
\noindent Two IEC standards were located with relevance to our device that relate to interruptions and fluctuations in voltage. The IEC 61000-4-11\cite{iec_standard_2} is a standard that requires components to endure electrical disturbances including short and long term changes or drops in voltage. We should use components that comply with these standards as there is not always a stable voltage provided from any power supply. An extreme example of the vitality of these standards is if the power went out during a hurricane while a laptop was connected to the outlet. Without some form of voltage regulation and durable parts, the laptop could be damaged. Likewise, even if there is simply noise coming in from the power supply, we do not want any components to be damaged from these fluctuations. More specifically, the IEC 61000-4-29 deals with electrical disruptions due to a DC power supply \cite{iec_standard_3}. These standards are more relevant to our product because our device is mobile. The walker must be able to move freely and travel long distances, so we will be using a rechargeable DC power supply. It would be beneficial to research components that meet the IEC 61000-4-29 standards so that damage is prevented from any predictable or unpredictable changes in voltage provided by the power supply.


\subsubsection{Quality Component Standards}
\noindent There is an Society of Automotive Engineers (SAE) standard that deals with counterfeit electrical parts \cite{sae5553c2019}. In order to maintain quality components, this standard SAE AS 5553C-2019 provides steps to minimize the risk of purchasing counterfeit components. Counterfeit parts could become an issue with our project if we purchase defective or poor quality components. The quality of the entire walker would not be the best and this could prove to cause a reliability or safety hazard.\\

\noindent There are also standards relating to the "robustness validation of semiconductor devices in automotive applications" described in SAE J 1879-2014. These standards again relate to the quality of components used in our design and provide a guideline for selecting quality, specifically reliable, components. These standards aim to have zero defects with components to increase reliability. We want our components to be reliable to reduce the risk of errors and have a long-lasting product. If a component fails while FORWARD is being deployed, this could cause an issue with safety.\cite{sae18792014}

% IEEE standards here
\subsubsection{Connection Standards}
\noindent USB Type C is notably used to connect the ESP32 Camera module to the central processor. In addition, we utilized USB Type B to program the Arduino Uno for initial sensor interfacing. There are unique connectors for the DC brushless motors as well. The LiDAR sensor utilizes a 6-pin Japan Solderless Terminal (JST).

\subsubsection{Computer Vision Standards}
\noindent The Artificial Intelligence Committee of IEEE published the IEEE P3110 standard which describes the algorithms for computer vision as well as the API model. These standards help regulate the interfaces between algorithms, frameworks, and sets of data to form a standardized API. This aids in the compatibility of computer vision software. \cite{ieeep3110}

% ANSI standards here