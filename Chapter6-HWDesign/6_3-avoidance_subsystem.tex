\subsection{Avoidance Subsystem}
% intro paragraph

% do we fix the wheels in place? most of them swivel like castors
\subsubsection{Wheel Rotational Motion}
\noindent The rollator wheels will dictate the way by which turning can be achieved. There are four wheels. We call the distance in between the front wheels $axel_f$ and the distance in between the rear wheels $axel_r$. Upon initial inspection of the rollator once acquired, we see that $axel_f > axel_r$. The measurements are as follows:\\

[TABLE HERE OF MEASUREMENTS]\\

\noindent From a top view, the rear two wheels can be viewed as simple points. They do not swivel, and thus can rotate in place but cannot do horizontal translation, stipulated by friction and attachment to the rollator frame. Of course, they can also roll and move forward.\\

\noindent In addition, the front wheels are on a $360^{\circ}$ swivel. They can also be modeled as points in a sense; however, their orientation changes

\noindent \underline{\textit{Case 1}} (extreme): both wheels rotate. The front wheels translate.\\

\noindent \underline{\textit{Case 2}} (extreme): one wheel rotates, the other stationary. The front wheels revolve.\\

\noindent \underline{\textit{Case 3}} Veering. motor speeds may not have to change. User can veer based on feedback, and gentle guidance provided by FORWARD.\\

% talking more about one wheel motor faster than the other. fusing user steering with FORWARD guidance.\
\subsubsection{Turning Mechanics}

% what methods and considerations do we have when sticking the motors on the rollator?
\subsubsection{Motor Installation}
% reconciling measurements acquired of the wheel diameter, weight, etc, with the torque of the DC motors.
\noindent \underline{\textit{Calibration}}

% does simply glueing the haptic motors on the handlebar underside sufficiently vibrate the handlebars to where the user notices? how much power would that require?
\noindent \underline{\textit{Vibrating Handlebars via Haptics}}

% talk about how the rollator has pneumatic handle squeeze brakes but braking can also be achieved by locking the DC motors
\subsubsection{Emergency Braking}

%issue about user trying to steer when they shouldn't
\subsubsection{Reconciling User Control and Automatic Guidance}
