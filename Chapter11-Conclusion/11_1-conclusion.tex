\noindent FORWARD is a form of autonomous rollator with the intent of helping users with various sensory impairments, particularly visual impairments, navigate the world around them. The initial motivation for our design was to provide an affordable solution for people who both are blind and have suffered injury or find it difficult to walk. FORWARD is supposed to provide to the user mobility aid, freedom, and safety to the extent possible.\\

\noindent Recall that there are three major subsystems to this device, including object detection, object identification, and object avoidance. Ultrasonic and LiDAR sensors will be used to detect obstacle on the side and front of the rollator. When an object comes within a close range, the camera module will use computer vision to classify what the object was that has come into range. This data will then be transmitted to give commands to the audio and haptic feedback, as well as the drive motors. The feedback provided to the user will allow them to be aware of their surroundings and the actions of the rollator. This is to prevent confusion or stumbling of the user. In addition, the walker will turn to the side to avoid obstacles detected based on the location of the obstacle.\\

\noindent Our design also includes several peripherals, including a speed dial, IMU stability, and a headlight. A potentiometer is used as a dial to select the initial walking speed of the rollator. The purpose of this dial is to provide freedom to the user to walk at a comfortable pace. Another peripheral that we added to the walker is an IMU (tilt sensor) that will be able to determine the angle of the walker on an incline. This will be integrated with the motors to prevent FORWARD from rolling backward whilst driving up a ramp or hill. Our design also incorporates a headlight that will illuminate the path in front of the user at night.\\

\noindent FORWARD has certainly been a learning experience for our team. We have been able to go in depth with the engineering design process throughout this project, from defining criteria and constraints to researching component technologies, designing the system integration, and finally to begin implementing what we have been planning all semester. Every step of this process in vital, not simply the design. It was imperative to first define what we expect form our final product and research each step of how to accomplish this.\\

\noindent We are now in a good position for the upcoming semester for Senior Design II. Our rollator is able to detect range using the ultrasonic and LiDAR sensors, identify some objects with the camera, and vibrate the haptic motors. We have also been able to integrate these subsytems together into one program, with the haptic feedback dependent on the sensor data. Our main goals for the next semester are to order/incorporate the PCB, implement the drive motors, fully integrate the control of wheels and feedback using camera and sensor input, properly mount components, and stress test our project.