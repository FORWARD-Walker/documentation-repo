\subsection{Project Motivations}
\indent This project has the ability to help those who are physically impaired as a result of surgery recovery or have a more permanent hindrance such as blindness or deafness. This is also especially useful for people who both have a visual impairment and have also suffered a physical injury. In addition, FORWARD is a much cheaper option than a wheelchair so that the users have an affordable way to get around. \\

\noindent At the core, conventional walkers do little to assist the user outside of bearing their weight and providing a barrier between them and any obstacles. The user is still susceptible to tripping and falling because of an obstruction the walker collides with. This is especially harmful for those with sensory impairments because, if they are injured, they will not be able to have the same mobility as before. For example, people who are blind use a white cane to feel the ground in front of them while walking, which is not feasible to do when both hands are being used to support their weight. \\

\noindent It will be a great learning experience to implement a variety of different technologies that we are not yet accustomed to, including computer vision, sensor fusion, and guidance systems. We will also be able to gain more practical experience to apply what we have learned in our classes about embedded systems, programming, and control systems. There are even aspects of linear circuits that we will be able to apply, such as power limitations between components and wiring a PCB. \\

\noindent With the technology available today, the opportunity exists to improve the quality of life and safety for walker users and enable them to have a faster and more secure lifestyle while traveling. Spend less time worrying about injury and more time enjoying the company of family and friends. \\