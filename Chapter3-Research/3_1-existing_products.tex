\subsection{Existing Products and Projects}
\noindent There have been numerous efforts made by researchers to develop smart walkers using the available technologies. In the following sections we seek to analyze these technologies in a way that identifies which of such will be most desired for our specific system, taking into  consideration many factors, some of which being: cost, attainability, integration complexity, and performance.\\

\noindent Why the need for 7 ultrasonics? in Mostofa \cite{Mostofa}\\

\noindent Autonomous robots are a popular use of LiDAR\\

\noindent \textbf{Image Processing Research}
\newline
 Ultralytics YOLO is an open source AI driven image processing model. The Ultralytics YOLO platform provides an AI Image processing Architecture that is malleable to the users demands, and for our case, would process the input image to detect if one of the five hazards is present, as well as provide a confidence rating for these predictions. To implement the object detection functionality to our specifications, minimal training, as well as coding must be performed. The benefits of this solution for our image classification system include zero cost, configurability, and minimal memory footprint (a few hundred MB) - which are all important aspects for our desired solution. The negatives include the needed man hours to train and implement this model for our purposes and time to run (anywhere from 0.1 seconds to 3 seconds). The time to predict could exclude this solution from our options as we need to have classification under 1 second latency, ideally in the realm of a few hundred milliseconds. 
%https://www.freecodecamp.org/news/how-to-detect-objects-in-images-using-yolov8/#heading-how-to-get-started-with-yolov8 
\newline 

\newline 
\noindent FOMO (Faster objects, More objects) is a machine learning image processing algorithm that seeks to solve the problem of running high complexity algorithms (such as neural networks) on MCU's. In the documents referenced [HERE], the Arduino Nicla board is used to classify the objects in frame at a rate of 33 microseconds (30 SPS). The objects can be classified and processed through multiple methods, but all of which would be applicable to our purposes. This system also minimizes the memory usage by needing only 256 KB of memory to run and store these algorithms. On the FOMO website, they claim that this algorithm is compatible with many of the MCU's that we are considering for this project, some of which being an ESP32, Arduino Nicla, and others.
%https://www.hackster.io/mjrobot/tinyml-made-easy-object-detection-with-nicla-vision-407ddd
%https://docs.edgeimpulse.com/docs/edge-impulse-studio/learning-blocks/object-detection/fomo-object-detection-for-constrained-devices
\newline 

