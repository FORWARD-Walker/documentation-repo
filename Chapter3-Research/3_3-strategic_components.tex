
\subsection{Strategic Components}
% this is pretty much all research about the components themselves, which is summed up with the spec comparison table in 3_5-partselect

\subsubsection{Ultrasonic Sensors} \label{sssec:tof}
\noindent The range solution according to \textbf{\textit{time-of-flight sensing}} is modeled by the simple equation, where $d$ is distance, $t$ is time to send and return an ultrasonic pulse, and $v$ which is the speed of sound (340m/s):
$$d = 1/2 \times t \times v$$

\noindent \underline{\textit{HC-SR04}} is probably the most popular ultrasonic sensor used by developers worldwide. Its ECHO pin outputs the time, $t$, taken by an emitted sound pulse (sent by a powered microcontroller) (TX) to return (RX). In practice, to continuously sense and report range, the microcontroller will repeatedly send 40kHz pulses to the sensor's TRIG pin, which are redirected by the transmitter as a beam. Using I2C, the microcontroller can read $t$ and perform the range calculation.\\

\noindent \underline{\textit{HRLV-MaxSonar MB1000}} This sensor features auto calibration, range filtering and many more output options: pulse width, analog voltage, ranging start/stop (real time), and serial output. The designers also claim that this higher quality sensor does not skew the range reading based on the target size as others do. To solve for range, either connect Pin 3 to an analog-to-digital converter and calculate:
$$d (mm) = bits_{\{0..1023\}} \times 5$$
Alternatively, Pin 4 gives options for real time range data based on its pull-up voltage. It can return a range on command by MCU or default to 2Hz filtered range data "based on recent ranges." Finally, Pin 5 is serial output in RS232 format, which sends in ASCII, the range in millimeters.\\

\noindent \underline{\textit{RCWL-1X0X}} sensors allow for the same solutions method as the HC-SR04 and MB1000. These models feature both UART and I2C serial outputs, in which the distance is read directly from the device using:\\
UART: $Distance = ((BYTE_H << 16) + (BYTE_M << 8) + BYTE_L) /1000$\\
I2C: $Distance = ((BYTE_H\&It; \&It; 16) + (BYTE_M\&It: \&It; 8)+BYTE_L) /1000$\\
\noindent During development, using libraries will likely circumvent the need to interact with the sensors at a low-level as shown above (bits and bytes). Later chapters will discuss methods used to test and begin to develop with the components once they are chosen.\\

\noindent \underline{\textit{H2KA150KA1CD00}} operates at the high frequency of 150kHz. Unfortunately, the datasheet does not reveal much more about solving the range than the need to use a rectangular wavwdrive signal and a burst of 10 pulses. This returns an analog voltage output of the range. At a directivity angle of 8+/-2 degrees and a short range viability, this is not an ideal option for FORWARD.\\

% account 1 page for ultrasonic table
\begin{table}[H]
	\centering
	\setlength{\tabcolsep}{5pt} % Default value: 6pt
	\renewcommand{\arraystretch}{1.5} % Default value: 1
	\begin{NiceTabular}{ccccccc}[hvlines,colortbl-like]
		\CodeBefore
		%\columncolor{}{}
		\rowcolor{maroon!50}{1}
		\cellcolor{}{4-1}
		\Body
		Model       & HC-SR04 & RCWL1XX5 & RCWL1670   & MB1000 & H2KA  & SEN0150 \\
		Type        & Multi   & XCVR     & Multi      & XCVR   & XCVR  & XCVR    \\
		Range (cm)  & 450     & 550      & 400        & 600    & 250   & 1000    \\
		Angle (deg) & +/-15   & +/-30    & +/-15      & +/-20  & 8+/-2 & +/-15   \\
		Accuracy    & 0.3 cm  & +/-2\%   & +/-2\%     & 1 inch & 2 cm  & 1 cm    \\
		Protocol    & I2C     & I2C      & I2C        & ?      & ?     & UART    \\
		Freq. (kHZ) & 40      & 40       & 40         & 42     & 150   & 49.5    \\
		Note        &         & deadzone & waterproof &        &       &         \\
		Price       & \$      & \$       & \$         & \$\$   & \$\$\$& \$\$\$
	\end{NiceTabular}
	\caption{\label{fig:compareUltrasonic}Ultrasonic Sensor Options}
\end{table}

\subsubsection{LiDAR Sensors}
\noindent \underline{\textit{LiDAR Lite V3}} is an optical distance sensor manufactured by Garmin (known for the GPS solutions). The listing specifies this sensor is great for applications in robotics and drones. They call it "single-stripe" with its aperture of only 12.5mm. The procedure for obtaining the reading is as follows:
\begin{enumerate}
	\item Write 0x04 to register 0x00.
	\item Read register 0x01. Repeat until bit 0 (LSB) goes low.
	\item Read two bytes from 0x8f (High byte 0x0f then low byte 0x10) to obtain the 16-bit measured distance in centimeters
\end{enumerate}

\noindent \underline{\textit{XT-S1}} is like the Garmin in that it is a more premium option at a higher price, but it does give a very long range of 30 meters. It allows for active continuous measurement of range by reading start address $0x00$. Its output is hexadecimal, which is easily converted to a decimal reading in millimeters. XT-S1 is specified for autnomous navigation of automobiles and UAV avoidance.\\

\noindent \underline{\textit{TF Family}} LiDAR sensors allow continuous ranging and also read by command. Note these have a 20cm dead zone. The designers note the possibility of light shedding on two parts of a surface, at which the range reading could be any value inbetween. They advise avoiding this for reliability. In the greater FORWARD context, we can consider the lesser of the readings for safety, so this is not necessarily an issue.\\

% account 1 page for lidar table
\begin{table}[H]
	\centering
	\setlength{\tabcolsep}{5pt} % Default value: 6pt
	\renewcommand{\arraystretch}{1.5} % Default value: 1
	\begin{NiceTabular}{cccccc}[hvlines,colortbl-like]
		\CodeBefore
		%\columncolor{}{}
		\rowcolor{maroon!50}{1}
		\cellcolor{}{4-1}
		\Body
		Model       & SEN-14032               & TFmini-S & TFmini Plus & TF-Luna & XT-S1   \\
		Type        & Single 				  & Single   & Single      & Single  & Single  \\
		Range       & 40m-70\% $\rho_\lambda$ & 0.1-12m  & 0.1-12m     & 0.2m~8m & 0.3-30m \\
		Angle (deg) & +/-0.5                  & +/-2     & +/-2.3      & +/-2    & +/-1.5  \\
		Accuracy    & <5m,+/-2cm              & <6m,1cm  & 1\%-2\%     & +/-6cm  & +/-0.1m \\
		Note        & nonlinearity <1m        &  &  &  & \\
		Price       & \$\$\$                  & \$\$     & \$\$        & \$\$    & \$\$
	\end{NiceTabular}
	\caption{\label{fig:compareLiDAR}LiDAR Sensor Options}
\end{table}

\subsubsection{Inertial Measurement Units}
\noindent Because of the simplistic nature of the application of this technology in FORWARD, extensive research is not as necessary as with the range sensors. We consider two main options.\\

\noindent \underline{\textit{BNO055}} is a 9-DOF (axis) IMU, meaning it contains a 3-axis accelerator, 3-axis gyroscope, and a 3-axis geomagnetic sensor. This device is said to run sensor fusion on-board to provide accurate attitude information. It runs on $2.4-3.6V$ and draws $0.4mA$ of current when on its low power mode. It can be used in 3-axis only modes, meaning you can isolate the accelerator, gyroscope or magnetometer.\\

\noindent \underline{\textit{MPU-6050}} is one of the most popular IMU devices. It is 6-DOF (no magnetometer), and runs a DMP (digital motion processor) for motion fusion algorithms. Essentially, it calibrates itself, and is said to reduce gyro drift by eliminating cross-axis alignment errors between the gyroscope and accelerator. It operates at $3.6mA$ with a supply of $2.4-3.5V$.\\

% IMU comparison
\begin{table}[H]
	\centering
	\setlength{\tabcolsep}{5pt} % Default value: 6pt
	\renewcommand{\arraystretch}{1.5} % Default value: 1
	\begin{NiceTabular}{ccc}[hvlines,colortbl-like]
		\CodeBefore
		%\columncolor{}{}
		\rowcolor{maroon!50}{1}
		\cellcolor{}{4-1}
		\Body
		Model         & BNO055 & MPU-6050 \\
		Axis          & 9-DOF   & 6-DOF    \\
		Sensitivity SF Tolerance & 1\% & 3\%  \\
		Nonlinearity  & 0.2\%   & 0.5\%    \\
		Sensor Fusion & Yes     & Yes      \\
		Price         & \$      & \$       \\
	\end{NiceTabular}
	\caption{\label{fig:compareIMU}IMU Sensor Options}
\end{table}

\subsubsection{Computer Vision/AI Image Processing Models}
\underline{\textit{Ultralytics YOLO}} (You only look once) is an open source AI driven image processing model. The Ultralytics YOLO platform provides an AI Image processing Architecture that is malleable to the users demands, and for our case, would process the input image to detect if one of the five hazards is present, as well as provide a confidence rating for these predictions. To implement the object detection functionality to our specifications, minimal training, as well as coding must be performed. The benefits of this solution for our image classification system include zero cost and easy configurability which are both important aspects for our desired solution. The negatives include the needed man hours to train and implement this model for our purposes and time to run (anywhere from 0.1 seconds to 3 seconds). It also has a relatively high memory footprint (a few hundred MB). The time to predict along with the memory footprint could potentially exclude this from our options as we need to have classification under 1 second latency, ideally in the realm of a few hundred milliseconds and enough storage space for the model. \\
%https://www.freecodecamp.org/news/how-to-detect-objects-in-images-using-yolov8/#heading-how-to-get-started-with-yolov8 

\noindent \underline{\textit{FOMO}} (Faster objects, More objects) is a machine learning image processing algorithm that seeks to solve the problem of running high complexity algorithms (such as neural networks) on MCU's. In the documents referenced [HERE], the Arduino Nicla board is used to classify the objects in frame at a rate of 33 microseconds (30 SPS). The objects can be classified and processed through multiple methods, but all of which would be applicable to our purposes. This system also minimizes the memory usage by needing only 256 KB of memory to run and store these algorithms. On the FOMO website, they claim that this algorithm is compatible with many of the MCU's that we are considering for this project, some of which being an ESP32, Arduino Nicla, Raspberry Pi, and others. \\
%https://www.hackster.io/mjrobot/tinyml-made-easy-object-detection-with-nicla-vision-407ddd
%https://docs.edgeimpulse.com/docs/edge-impulse-studio/learning-blocks/object-detection/fomo-object-detection-for-constrained-devices

\noindent \underline{\textit{Edge Impulse}} is a website that is free to use and will allow for you to train models and then generate code for specific target devices. The way it works is, you upload your images for object detection, label and frame the objects, and then train the model. After this, you can choose your target device (Edge Impulse supports RaspberyrPi, ESP32, Arduino, and more), and generate a zip file for a library of object detection code. This is a simple to use, low complexity option that supports all of our considered MCUs.\\
%https://edgeimpulse.com/
%https://www.youtube.com/watch?v=bZIKVaD3dRk\

\begin{figure}[H]
	\centering
	\setlength{\tabcolsep}{5pt} % Adjust column spacing
	\renewcommand{\arraystretch}{1.75} % Adjust row spacing
	\scalebox{1.0}{ % Scale down the entire table by 85%
		\begin{NiceTabular}{lccc}[hvlines,colortbl-like]
			\CodeBefore
			\rowcolor{blizzardblue}{1} % Row background color for header
			\Body
			Specification             & Ultralytics YOLO           	& FOMO                    		& Edge Impulse \\
			\hline
			Processing Time           & 6-100 ms                   	& 10-40 ms            			& 10-50 ms \\
			Memory Usage              & Few hundred MB             	& 256 KB                   		& 100 KB-1 MB \\
			Training Complexity       & Low        			& Moderate                      & High \\
			
			
			
		\end{NiceTabular}
	}
	\caption{\label{fig:aiComparison}Comparison of AI Image Processing Solutions}
\end{figure}


\subsubsection{Camera Technologies}
\label{sec:cam_tech}
\noindent \underline{\textit{ESP 32 CAM Board}} is a ESP32 processor based camera module that can transmit video data at high resolutions, which can be processed by our CV/Image processing solution. The module also has WIFI connectivity that can connect to an MCU to wirelessly transmit the video stream. The module has a 32-bit CPU with a max clock frequency of 240 MHz and 520 KB of built-in SRAM with an external 4MB PSRAM. The board module also runs on FreeRTOS. The included camera, OV2640 camera, has a 2 MP resolution, up to 1600 x 1200 and interfaces the controller board over a 24 pin interface bus. A more advanced sister product to the OV2640 is the OV5640 camera. This camera has 5 MP resolution, up to 2592 x 1944. In order to achieve our stretch requirement of depth perception analysis for guidance and navigation, this camera could be advantageous.\\

\noindent \underline{\textit{Oak-D Lite}} is an AI robotics specific camera, capable of running AI models such as open CV. The camera is able to run on-board AI architectures to achieve object classification, edge detection, and feature tracking. This technology allow for high accuracy with object detection, and introduce the ability to implement edge detection as a means to depth perception, which can be used for routing in our GNC subsystem. The Oak-D Lite has a 13 MP RGB camera with auto-focus, as well a 60axis sensor using an accelerometer and gyroscope. It also uses USB2 and USB3 for power and communication, which is simple to interface to a RaspberryPI. The camera can run 4K at 30 FPS or 1080P at 60 FPS.\\

\noindent \underline{\textit{Realtek AMB82-Mini IoT AI Camera}} is a lower cost option (\$25) to incorporate an AI object detection system within the camera and board. The specifications of the device are as follows: The MCU is an ARMv8M running up to 500 MHz, accompanied by an NPU with an Intelligent Engine capable of 0.4 TOPS. It features 128 MB of internal DDR2 memory. Connectivity options include dual-band Wi-Fi and Bluetooth 5.1.The audio code supports ADC/DAC/I2S, and the ISP/Video functionality includes both 1080p and 720p at 30FPS. The camera module is a JXF37 full HD CMOS image sensor with a resolution of 1920 x 1080 and a wide-view-angle FOV of 130°. The interface includes 1 microphone, 2 Micro USB B ports, 1 MicroSD card slot, and support for 3 UARTs, 2 SPIs, 1 I2C, 8 PWMs, 2 GDMAs, and a maximum of 23 GPIOs. This solution would free up processing power on our MCU which could be of great value given the high workload placed on our MCU including sensor fusion, motor controlling, image processing, and other needed functionalities. Also, due to the high memory capacity (128 MB DDR2), we could implement a higher quality detection model.\\

% Camera comparison
\begin{figure}[H]
	\centering
	\setlength{\tabcolsep}{5pt} % Adjust column spacing
	\renewcommand{\arraystretch}{1.75} % Adjust row spacing
	\scalebox{0.8}{ % Scale down the entire table by 85%
		\begin{NiceTabular}{lccc}[hvlines,colortbl-like]
			\CodeBefore
			\rowcolor{blizzardblue}{1} % Row background color for header
			\Body
			Specification             & ESP32 CAM                  & Oak-D Lite                       &  AMB82-Mini\\
			\hline
			Resolution                & 2MP/5MP   & 13 MP (RGB)                      & 1920 x 1080 (Full HD)            \\
			Frame Rate                & 1600x1200/2592x1944       & 4K@30FPS/1080P@60FPS     & 1080P/720P@30FPS           \\
			CPU/MCU                   & ESP32                & RVC2         & ARMv8M   \\
			Memory                    & 520 KB SRAM, 4 MB PSRAM         & 512 MB LPDDR4 RAM                             & 128 MB DDR2                     \\
			Interface                 & Wi-Fi/24-pin Cam Bus        & USB2/USB3                        & WiFi/BT/USB/GPIO    \\
			Price                     & \$10-15                 		& \$149                			& \$25-\$30                     \\
		\end{NiceTabular}
	}
	\caption{\label{fig:compareCameras}Comparison of Camera Modules for ESP32-based Applications}
\end{figure}


\subsubsection{MCU Technologies}
\label{sec:mcu_tech}
\noindent The Embedded Processor/MCU for the project will be the "brains" of the system. It will bring together every other subsystem cohesively and run all necessary computations, while needing to maintain a minimal footprint in aspects such as power and physical size. We must also consider the cost effectiveness. It will also need I2C communication buses to interface the Sonar and LiDAR Sensors, Bluetooth to communicate with the audio feedback device, WiFi to communicate with the Camera, and UART/SPI/I2C to interface with the motor drivers/controllers. Given all of these needs and constraints, we will be looking into the current state of MCU products. \\

\noindent \underline{\textit{The ESP32 MCU}} is a small but powerful microcontroller with many features. These features include UART, SPI, and I2C communications, 4 MB - 8 MB of flash, with some PSRAM options. These boards also have WIFI and Bluetooth connectivity. The ESP32 series of MCU's range anywhere from \$9 to \$12. The current existing boards are ESP32-DevKitC-32E, ESP32-DevKitC-32UE, ESP32-DevKitCVE, ESP32-DevKitCVIE, and ESP32-DevKitCS1. The main difference between these boards are size and types of memory (4 MB - 8 MB and PSRAM/FLASH) along with antenna type (IPEX vs PCB). The ESP32 family also has a variety of products, such as camera modules, motor controllers, and others. The ESP32 is also easy to program as it is compatible with the Arduino IDE. \newline

\noindent The final consideration for the ESP-32-WROOM DevKitC is which of the variation would best fit our purposes, the standard, U, or D series. The D contains a differen PCB design relative to the standard which improves RF performance. The U is optimized for wider range WiFi capabilities due to an external IPEX antenna. The cost is slightly different coming in at \$6.67, \$6.67, and \$7.50 respectively. The D is the best fit for our purposes given we don't need wide range WiFi since the system is within a cubic meter in size, but the RF performance improvement of the D would be helpful given we will be transmitting our camera data over WiFi. \\
%https://www.espressif.com/en/products/devkits/esp32-devkitc

\noindent \underline{\textit{The RaspberryPI 4 model B}} is a small computer with a 64-bit quad core processor running at 1.5GHz. The RaspberryPi has both dual-band 2.4GHz and 5GHz wireless LAN and Bluetooth 5.0/BLE. The Pi also has Ethernet capabilities along with a 40 pin GPIO header. It uses the Broadcom BCM2711 processor chip and comes with memory options of 1GB, 2GB, 4GB or 8GB LPDDR4-3200 SDRAM, as well as having a micro-SD slot for additional memory capabilities. The board also features many I/O ports, such as 2 USB 3.0 ports, 2 USB 2.0 ports, 2 HDMI ports, and a 2-lane MIPI CSI camera port. The board ranges from \$30 to \$50 and can be purchased online. \\
%https://www.pishop.us/product/raspberry-pi-4-model-b-2gb/?src=raspberrypi

\noindent \underline{\textit{The Jetson Nano}} is a complex, high powered MCU designed for AI/ML applications. It is an NVIDIA product with a 128-core NVIDIA Maxwell™ GPU and a quad-core Arm® A57 CPU. It comes with 4 GB of 64-bit LPDDR4 memory and supports MicroSD storage. For power, it supports both 5 V, 4 A DC power and 5 V, 2 A via Micro-USB. In terms of connectivity, it offers a USB 3.0 Type-A port, a USB 2.0 Micro-B port, an HDMI/DisplayPort output, and Gigabit Ethernet. The board also includes various general-purpose input/output (GPIO) options, along with interfaces for I2C, I2S, SPI, UART, and a MIPI-CSI camera connector. These features make the Jetson Nano a robust solution for real-time processing tasks needed for guiding and navigating. This board is designed for running AI/ML applications. The MCU can cost anywhere from \$250 to \$500 depending on the model and specifications.\\

% MCU comparison
\begin{figure}[H]
	\centering
	\setlength{\tabcolsep}{5pt} % Restore default column padding
	\renewcommand{\arraystretch}{1.75} % Restore default row height
	\scalebox{0.85}{ % Scale down the entire table by 85%
		\begin{NiceTabular}{cccccc}[hvlines,colortbl-like]
			\CodeBefore
			\rowcolor{blizzardblue}{1} % Row background color for header
			\Body
			Model            & ESP32D & Raspberry Pi 4 B & Jetson Nano \\
			\hline
			Processor        & ESPWROOM     & Broadcom BCM2711 & NVIDIA Maxwell GPU \\
			Memory           & 4MB-8MB Flash/PSRAM  & 1GB-8GB LPDDR4   & 4GB LPDDR4 \\
			Connectivity     & WIFI, Bluetooth      & Wi-Fi, Bluetooth, Eth. & Ethernet \\
			I/O Ports        & GPIO, UART, I2C, SPI & GPIO, 2 USB 3.0, 2 HDMI & USB: 3.0, 2.0, HDMI \\
			Price            & \$5 - \$10           & \$30 - \$50       & \$250 - \$500 \\
		\end{NiceTabular}
	}
	\caption{\label{fig:mcuComparison}Comparison of MCU Options}
\end{figure}


\subsubsection{Bone Conduction Headphones}
\noindent As discussed above, we have decided to use bone conduction headphones as our audio feedback device due to the light weight, low cost, high performance aspects of this solution. Listed below are the specific products we considered as well as their specifications. The main specifications of the products that we will be considering is cost and battery life, with a secondary emphasis on comfort. \\

\noindent \underline{\textit{YouthWhisper}} headphones are the first ones considered for our system. They are rated for waterproof IP54 which means that they are "water and splash resistant" which is desired for if our system was to be used in the rain. It is also rated for 8 hour battery life  and a two hour charge time that charges with microUSB. They also have a 20-hour standby time, so plenty of time for the user to have for a full day. The cost \$35 and can be found at most retail stores. Due to their cheaper price point, comfortability can be an issue for longer periods of usage. They weigh in at 25 grams. \\

\noindent \underline{\textit{AfterShokz}} is the second product considered for our system. The main draw of these is that, although they have a price point of around \$130, we can implement them for free due to one of our project members already owning a new pair of them. They are rated at IP67 which is a high grade of water resistance, also suitable for rainy environments. AfterShokz also have an 8 hour battery life but have a one hour charge time that charges with the include power cord. They also have a 20-hour standby, which is suitable for full day use. They weigh in at 26 grams. \\

% Headphones comparison
\begin{figure}[H]
	\centering
	\setlength{\tabcolsep}{5pt} % Restore default column padding
	\renewcommand{\arraystretch}{1.75} % Restore default row height
	\scalebox{1}{ % Scale down the entire table
		\begin{NiceTabular}{ccccc}[hvlines]
			\CodeBefore
			\rowcolor{blizzardblue}{1} % Row background color for header
			\Body
			Model             & AfterShokz  & YouthWhisper \\
			\hline
			Battery Life      & 10 hours    & 8 hours       \\
			Charging Time     & 1 hour      & 2 hours       \\
			Water Resistance  & IP67        & IPX54         \\
			Weight            & 26 grams    & 25 grams      \\
			Price             & FREE        & \$35          \\
		\end{NiceTabular}
	} % End scalebox
	\caption{\label{fig:headphonesComparison}Comparison of Headphone Options}
\end{figure}


\subsubsection{Motor Controllers}
\noindent When evaluating motor controllers, there are several factors to consider and are largely connected to our decision of motors. The voltage range of the selected motor controller must be able to provide voltage as low as 5V and as high as 21.6V as well as provide a current of at least 11A. This ensures that both types of motors we selected are able to function properly.\\

\noindent \underline{\textit{UMLIFE}} is a high voltage motor controller with a large rated current. This controller uses pulse width modulation to operate the motors and has a quick delivery time. However, it is only able to control one motor. In the design of our FORWARD product, we will need a total of four DC motors: two motors to drive and guide the wheels, and two motors to provide haptic feedback. Thus, it is desired that we select a motor controller or combination of motor drivers that are able to control all four motors.\\

\noindent \underline{\textit{MKSESP32FOC}} is a motor controller that allows for 2 DC motors to be operated at one time and can either utilize I2C or SPI communication. However, this controller only provides a limited voltage range of 12-24V and can only support a current draw of up to 6A per channel. The current draw is an issue because the DC motors that we selected have a maximum current draw of 11A. Likewise, because of our motor selection of haptic motors (3-5V), the MKSESP32FOC would not be able to support the haptic motors.\\

\noindent \underline{\textit{768861A\_Y1181}} is a motor controller that operated in the range of 5-30V, ensuring that it can be utilized both for haptics motors and wheel-driving motors. This controller also provides 4 DC motors supplied with up to 10A of current per channel. This is less than the 11A maximum current drawn by the selected motors, however, upon examining the data sheet, this driver allows for somewhat larger current than the rated 10A to flow, with some degradation of the lifespan of the controller, as there was a graph labeled Life expectancy vs. Current of Load. If the motors draw up to 12A of current, the expected life expectancy according to this graph would be 80,000 operations. This should not be too much of an issue as the 11A drawn by the DC motors do not operate continually at 11A; it is simply a peak or maximum current. Although this may be seen as a reliability issue if used in industry, for our purposes, 80,000 operations should be far more than enough.  Overall, this is the controller that would best fit our design as all of the motors can be controlled from one board, reducing the cost, and all of the motors are able to be driven within the correct voltage and current range.\\

\noindent \underline{\textit{ESP32FOC}} is a motor controller that provides plenty of current: 20A. However, this controller can only power one motor at a time, thus requiring a total of four motor controllers. This would greatly increase the cost. The voltage range of 12V-28V is also not suitable for controlling the haptic motors.\\

\noindent \underline{\textit{TB6612}} is a motor controller that is lower voltage, with a range from 5-12V. This would work sufficiently for operating the haptic motors, however, not for operating the motors driving the wheels. This controller also only supplies 3A of current per channel, which again would be sufficient for the haptic motors, but the other motors draw too much current. The benefit of this controller would be that it can operate up to four motors at once.\\

\noindent  For our purposes, at first glance, it would be ideal to invest in a motor controller. This is because a motor controller would be simpler to program and more risks are taken care of at the controller level. We also need to use multiple drivers since we will need to control a number of motors. However, one of the constraints for this course is significant PCB design. This means that we have a limit on the number of development boards that can be integrated into the PCB. Many motor controllers are development boards, meaning that they include an MCU already integrated into a pre-designed PCB. Thus, in order to meet the requirement of significant PCB design, we need to evaluate if there is a suitable motor controller that does not use a development board. If we are unable to find such a motor controller, it will be necessary to implement the use of motor drivers and integrate them into our PCB design. After comparing our options, the motor controller that we to be found most suitable for the needs of the DC motors is the 76881A\_Y1181, which employs a development board. We are limited to the number of development boards and discouraged from using them unnecessarily, therefore, we plan to use motor drivers instead that will control each motor individually. In addition, the inclusion of another development board could increase complexity as it must communicate with the main ESP32 over Wi-fi.\\

\subsubsection{Motor Drivers}
\noindent Evaluating motor drivers is a very similar process to evaluating motor controllers. The main criteria we need to look at are a) voltage matching, b) minimum current, and c) number of motors controlled. However, for criteria c), it is most typical for motor drivers to only drive one motor at a time, so we will likely need to buy multiple motor drivers at a time. Also, since the haptic motors will be controlled separately from the wheel-driving motors, we can better match the voltage and current requirements to each motor.\\

\noindent \underline{\textit{B09P6D5TMV}} is a low voltage motor driver for the purpose of driving the haptic motors. This driver has a large range of 1.8-15V, well encompassing the 3-5V range of the haptic motors. This driver also supplies 2A of current, which is far above the necessary 15mA. However, there are less resources available for this driver compared to other drivers as it is less commonly used.\\

\noindent \underline{\textit{L298N}} is a very large voltage-range motor driver (from 5-35V) that will be able to allow for the haptic motors to be operated. 2A of current are supplied from this motor driver, which is sufficient for the haptic motors, and this driver also allows for two motors to be driven from the same driver. The L298N is also the most cost-effective option and the most widely used driver out of the haptic motor drivers, providing for plenty of documentations and resources.\\

\noindent \underline{\textit{DRV8871}} is a motor driver that has a large voltage range of 6.5-45V, however, this is outside of the allowed range for our selected haptic motors. Although it supplies plenty of current for the haptic motors (3.6A), it is not enough for the wheel-driving motors and thus we cannot use this option for either type of motor. This driver is also least cost-effective haptic driver.\\
 
\noindent \underline{\textit{RRPDMSCSGSPC}} is a motor driver with a large voltage range of 12-40V, enveloping the necessary value of 21.6V for the selected DC motors. This option also supplies a maximum current of 10A, which may or may not be enough current supplied to the motors. If the motors draw their maximum current of 11A, it is possible for this driver to be damaged. This driver also only controls one motor, so it is not the most cost-efficient.\\
 
\noindent \underline{\textit{BTS7960}} is a motor driver that is rated or 27V, above our necessary 21.6V motor. Although this driver does not provide a range, typically the voltage range is centered around the rated voltage. When it comes to current, this driver provides the maximum amount of current out of any of the drivers. Although it comes at a slightly higher cost, the BTS7960 comes in a package of two, which is needed to provide for both motors and thus is net most cost-effective. Thus we decided to select this driver.\\
 
\noindent \underline{\textit{RR-6-90V-15A-PSC}} is a motor driver with the widest voltage range of 6-90V. This driver leaves no question if the 21.6V motors are able to be driven. The rated current for this driver also covers 15A of current, which is 4A above needed. However, this driver only comes in a package of one, and thus purchasing two drivers will come at double the price of \$9.99.\\
